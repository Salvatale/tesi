\chapter{Appendice: Codice sorgente}
In questa appendice e' riportato il codice sorgente utilizzato nel progetto.
\section{Codice in Python del server}
\begin{lstlisting}[style=pythonstyle,caption={Codice del server.py}, label={lst:server}]
from flask import Flask, json, request, jsonify
from chatbot import process_input
from mail_chatbot import send_email
import vector_db_maker as vector
import directory_manager as dm
app = Flask(__name__)
@app.route('/upload', methods=['OPTIONS', 'POST'])
async def upload_file():
    if request.method == 'OPTIONS':
        # Gestisci la richiesta OPTIONS
        response = app.make_default_options_response()
        # Aggiungi i metodi consentiti nella risposta
        response.headers['Access-Control-Allow-Methods'] = 'POST'
        # Aggiungi gli header consentiti nella risposta
        response.headers['Access-Control-Allow-Headers'] = 'Content-Type'
    elif request.method == 'POST':
        directory_name = "./data/"
        if 'jsonFile' in request.files :
            json_data = request.files['jsonFile'].read()
            json_dict = json.loads(json_data)
            directory_name += json_dict['userID'] + '/' + json_dict['directory_name']
        if 'files' not in request.files:
            return jsonify({'error': 'No file part'}), 400
        uploaded_files = request.files.getlist("files")
        file_names = []
        for file in uploaded_files:
            file.save(file.filename)
            file_names.append(file.filename)
        await vector.get_text(uploaded_files, directory_name)
        response = jsonify({'message': 'Files uploaded successfully', 'files': file_names})
    # Aggiungi gli header CORS alla risposta per consentire le richieste da origini diverse
    response.headers['Access-Control-Allow-Origin'] = '*'  # Cambia '*' con l'origine desiderata
    response.headers['Access-Control-Allow-Headers'] = 'Content-Type'
    return response
@app.route('/create_directory', methods=['OPTIONS', 'POST'])
def handle_create_directory():
    if request.method == 'OPTIONS':
        # Gestisci la richiesta OPTIONS
        response = app.make_default_options_response()
        # Aggiungi i metodi consentiti nella risposta
        response.headers['Access-Control-Allow-Methods'] = 'POST'
        # Aggiungi gli header consentiti nella risposta
        response.headers['Access-Control-Allow-Headers'] = 'Content-Type'
    elif request.method == 'POST':
        # Gestisci la richiesta POST
        data = request.json  # Ottieni i dati JSON dalla richiesta
        # Fai qualcosa con i dati ricevuti, ad esempio, restituisci una risposta
        user_id = data['userId']
        dm.create_user_directory(user_id)
        response = jsonify({'message': "user directory created"})
    else:
        # Se il metodo non e' OPTIONS o POST, restituisci un errore
        response = jsonify({'error': 'Metodo non supportato'})
        response.status_code = 405  # Metodo non consentito
    # Aggiungi gli header CORS alla risposta per consentire le richieste da origini diverse
    response.headers['Access-Control-Allow-Origin'] = '*'  # Cambia '*' con l'origine desiderata
    response.headers['Access-Control-Allow-Headers'] = 'Content-Type'
    return response
@app.route('/directory', methods=['OPTIONS', 'POST'])
def handle_request_directory():
    if request.method == 'OPTIONS':
        # Gestisci la richiesta OPTIONS
        response = app.make_default_options_response()
        # Aggiungi i metodi consentiti nella risposta
        response.headers['Access-Control-Allow-Methods'] = 'POST'
        # Aggiungi gli header consentiti nella risposta
        response.headers['Access-Control-Allow-Headers'] = 'Content-Type'
    elif request.method == 'POST':
        # Gestisci la richiesta POST
        data = request.json  # Ottieni i dati JSON dalla richiesta
        # Fai qualcosa con i dati ricevuti, ad esempio, restituisci una risposta
        user_id = data['userId']
        directory_list = dm.get_user_directory(user_id)
        response = jsonify({'directory_list': directory_list})
    else:
        # Se il metodo non e' OPTIONS o POST, restituisci un errore
        response = jsonify({'error': 'Metodo non supportato'})
        response.status_code = 405  # Metodo non consentito
    # Aggiungi gli header CORS alla risposta per consentire le richieste da origini diverse
    response.headers['Access-Control-Allow-Origin'] = '*'  # Cambia '*' con l'origine desiderata
    response.headers['Access-Control-Allow-Headers'] = 'Content-Type'
    return response
@app.route('/', methods=['OPTIONS', 'POST'])
def handle_request():
    if request.method == 'OPTIONS':
        # Gestisci la richiesta OPTIONS
        response = app.make_default_options_response()
        # Aggiungi i metodi consentiti nella risposta
        response.headers['Access-Control-Allow-Methods'] = 'POST'
        # Aggiungi gli header consentiti nella risposta
        response.headers['Access-Control-Allow-Headers'] = 'Content-Type'
    elif request.method == 'POST':
        # Gestisci la richiesta POST
        data = request.json  # Ottieni i dati JSON dalla richiesta
        # Fai qualcosa con i dati ricevuti, ad esempio, restituisci una risposta
        path_directory = "./data/" + data['userId'] + "/" + data['context']
        response_data = process_input(data['message'],path_directory)
        if data['function'] == 'message':
            response = jsonify({'message': response_data})
        elif data['function'] == 'mail':
            send_email(response_data,data['mail'],data['message'])
            response = jsonify({'message': 'mail sent'})
    else:
        # Se il metodo non e' OPTIONS o POST, restituisci un errore
        response = jsonify({'error': 'Metodo non supportato'})
        response.status_code = 405  # Metodo non consentito
    # Aggiungi gli header CORS alla risposta per consentire le richieste da origini diverse
    response.headers['Access-Control-Allow-Origin'] = '*'  # Cambia '*' con l'origine desiderata
    response.headers['Access-Control-Allow-Headers'] = 'Content-Type'
    return response
@app.route('/url', methods=['OPTIONS', 'POST'])
async def handle_url_request():
    if request.method == 'OPTIONS':
        # Gestisci la richiesta OPTIONS
        response = app.make_default_options_response()
        # Aggiungi i metodi consentiti nella risposta
        response.headers['Access-Control-Allow-Methods'] = 'POST'
        # Aggiungi gli header consentiti nella risposta
        response.headers['Access-Control-Allow-Headers'] = 'Content-Type'
    elif request.method == 'POST':
        # Gestisci la richiesta POST
        data = request.json  # Ottieni i dati JSON dalla richiesta
        # Fai qualcosa con i dati ricevuti, ad esempio, restituisci una risposta
        path_directory = "./data/" + data['userId'] + "/" + data['context']
        urls = str(data["message"])
        print(urls)
        await vector.urls_vectordb_maker(urls,path_directory)
        response = jsonify({'message': "Vectordb created successfully"})
    else:
        # Se il metodo non e' OPTIONS o POST, restituisci un errore
        response = jsonify({'error': 'Metodo non supportato'})
        response.status_code = 405  # Metodo non consentito
    # Aggiungi gli header CORS alla risposta per consentire le richieste da origini diverse
    response.headers['Access-Control-Allow-Origin'] = '*'  # Cambia '*' con l'origine desiderata
    response.headers['Access-Control-Allow-Headers'] = 'Content-Type'
    return response
if __name__ == '__main__':
    app.run(debug=True,port=8000)
\end{lstlisting}
\begin{lstlisting}[style=pythonstyle,caption={Codice del chatbot.py}, label={lst:chatbot}]
from langchain_community.vectorstores import Chroma
from langchain_community import embeddings
from langchain_community.llms import Ollama
from langchain_core.runnables import RunnablePassthrough
from langchain_core.output_parsers import StrOutputParser
from langchain_core.prompts import ChatPromptTemplate
import os
def process_input(question,persist_directory):
    if not os.path.exists(persist_directory):
        return "Invalid context, insert a valid context!!"
    db = Chroma(persist_directory=persist_directory, embedding_function=embeddings.OllamaEmbeddings(model='nomic-embed-text'))
    retriever = db.as_retriever()
    model_local = Ollama(model="mistral")
    #perform the RAG 
    question = "rispondi in italiano a: " + question
    after_rag_template = """Answer the question based only on the following context:
    {context}
    Question: {question}
    """
    after_rag_prompt = ChatPromptTemplate.from_template(after_rag_template)
    after_rag_chain = (
        {"context": retriever, "question": RunnablePassthrough()}
        | after_rag_prompt
        | model_local
        | StrOutputParser()
    )
    return after_rag_chain.invoke(question)
\end{lstlisting}
\begin{lstlisting}[style=pythonstyle,caption={Codice del mail\_chatbot.py}, label={lst:mailchatbot}]
import smtplib
from email.mime.text import MIMEText
def send_email(response,receiver_mail,question):
        server = smtplib.SMTP("smtp.gmail.com", 587)
        server.starttls()
        server.login("alle.salva7@gmail.com","**************")
        text = "Question: " + question + "\n" + "Risposta: " + response
        msg = MIMEText(text, _charset='utf-8')
        msg['From'] = "alle.salva7@gmail.com"
        msg['To'] = receiver_mail
        msg['Subject'] = "CHATBOT response"
        server.sendmail(msg['From'],msg['To'],msg.as_string())
        server.quit()
\end{lstlisting}
\begin{lstlisting}[style=pythonstyle,caption={Codice del directory\_manager.py}, label={lst:directorymanager}]
import os
def get_user_directory(user_id):
    directory = "./data/" + user_id
    elements = os.listdir(directory)
    directory_list = [element for element in elements if os.path.isdir(os.path.join(directory,element))]
    return directory_list
def create_user_directory(user_id):
    path_directory = "./data/" + user_id
    os.mkdir(path_directory)
\end{lstlisting}
\begin{lstlisting}[style=pythonstyle,caption={Codice del url\_parser.py}, label={lst:urlparser}]
import requests
from bs4 import BeautifulSoup
from urllib.parse import urlparse, urljoin
def get_same_domain_links(url):
    # Ottieni il contenuto HTML della pagina
    response = requests.get(url)
    # Controlla che la richiesta sia andata a buon fine
    if response.status_code == 200:
        # Analizza il contenuto HTML
        soup = BeautifulSoup(response.content, 'html.parser')
        # Ottieni il dominio base dell'URL passato
        base_domain = urlparse(url).netloc
        # Trova tutti i tag 'a' che contengono i link
        links = soup.find_all('a', href=True)
        # Lista per salvare i link con la stessa radice dell'URL passato
        same_domain_links = []
        for link in links:
            # Ottieni l'URL completo del link
            href = link['href']
            # Unisci l'URL relativo con l'URL di base per ottenere l'URL completo
            full_url = urljoin(url, href)
            # Controlla se l'URL appartiene allo stesso dominio dell'URL passato
            if urlparse(full_url).netloc == base_domain:
                same_domain_links.append(full_url)
        return same_domain_links
    else:
        # Se la richiesta non e' andata a buon fine, restituisci una lista vuota
        return []
\end{lstlisting}
\begin{lstlisting}[style=pythonstyle,caption={Codice del vector\_db\_maker.py}, label={lst:vector_db}]
    from PyPDF2 import PdfReader
import os
from bs4 import BeautifulSoup
from langchain_community.vectorstores import Chroma
from langchain_community import embeddings
from langchain.text_splitter import CharacterTextSplitter
from langchain_community.document_loaders import WebBaseLoader
import requests
from urllib.parse import urlparse, urljoin
async def get_text(pdfs, directory_name):
    text = ""
    for pdf in pdfs:
        pdf_reader = PdfReader(pdf.filename)
        for page in pdf_reader.pages:
            text += page.extract_text()
        os.remove(pdf.filename)
    text_splitter = CharacterTextSplitter(
        separator="\n",
        chunk_size=3000,
        chunk_overlap=1000,
        length_function = len
    )
    chunks = text_splitter.split_text(text)
    Chroma.from_texts(
        texts=chunks,
        persist_directory=directory_name,
        embedding=embeddings.OllamaEmbeddings(model='nomic-embed-text'),
    )
def get_all_link(url):
    # Ottenere il contenuto HTML della pagina
    response = requests.get(url)
    # Verifica se la richiesta ha avuto successo
    if response.status_code == 200:
        # Parsing del contenuto HTML
        soup = BeautifulSoup(response.content, 'html.parser')
        base_domain = urlparse(url).netloc
        # Estrazione di tutti i tag 'a' (link)
        links = soup.find_all('a',href=True)
        # Lista per memorizzare gli URL dei link
        all_links = []
        for link in links:
            # Ottenere l'attributo 'href' degli elementi 'a'
            href = link.get('href')
            full_url = urljoin(url,href)
            # Verifica se l'URL e' valido
            if urlparse(full_url).netloc == base_domain:
                # Aggiungi l'URL all lista dei link
                all_links.append(full_url)
        return all_links
    else:
        # Se la richiesta non ha successo, stampa un messaggio di errore
        print("Errore durante la richiesta HTTP:", response.status_code)
        return []
def get_all_urls(urls_list):
    urls = []
    for url in urls_list:
        urls.append(url)
        links = get_all_link(url)
        urls.extend(links)
    return urls
def get_content_from_url(url):
    try:
        content = WebBaseLoader(url).load()
    except Exception:
        return None
    return content
async def urls_vectordb_maker(urls,directory_name):
    docs = []
    # Convert string of URLs to list
    urls_list = urls.split("\n")
    urls_list= get_all_urls(urls_list)
    for url in urls_list:
        content = get_content_from_url(url)
        if content:
            docs.append(content)
    #docs = [WebBaseLoader(url).load() for url in urls_list]
    docs_list = [item for sublist in docs for item in sublist]
    #split the text into chunks
    text_splitter = CharacterTextSplitter.from_tiktoken_encoder(chunk_size=7500, chunk_overlap=100)
    doc_splits = text_splitter.split_documents(docs_list)
    #convert text chunks into embeddings and store in vector database
    Chroma.from_documents(
        documents=doc_splits,
        persist_directory=directory_name,
        embedding=embeddings.OllamaEmbeddings(model='nomic-embed-text'),
    )
\end{lstlisting}
\section{Codice in dart dell'applicazione Flutter}
\subsection{Pagine}
\begin{lstlisting}[style=pythonstyle,caption={Codice di main.dart}, label={lst:main}]
import 'package:chatbot/services/auth/auth_gate.dart';
import 'package:flutter/material.dart';
import 'package:chatbot/themes/light_mode.dart';
import 'package:firebase_core/firebase_core.dart';
import 'package:chatbot/firebase_options.dart';
Future<void> main() async {
  WidgetsFlutterBinding.ensureInitialized();
  await Firebase.initializeApp(options: DefaultFirebaseOptions.currentPlatform);
  runApp(const MyApp());
}
class MyApp extends StatelessWidget {
  const MyApp({super.key});
  // This widget is the root of the application.
  @override
  Widget build(BuildContext context) {
    return MaterialApp(
      debugShowCheckedModeBanner: false,
      home : const AuthGate(),
      theme: lightMode,
    );
  }
}
\end{lstlisting}
\begin{lstlisting}[style=pythonstyle,caption={Codice del login\_page.dart}, label={lst:login}]
import "package:chatbot/services/auth/auth_service.dart";
import "package:chatbot/components/my_button.dart";
import "package:chatbot/components/my_textfield.dart";
import "package:chatbot/services/user/user_string_list.dart";
import "package:flutter/material.dart";
class LoginPage  extends StatelessWidget{
  //email and pw text controllers
  final TextEditingController _emailController = TextEditingController();
  final TextEditingController _pwController = TextEditingController();
  // tap to go register page
  final void Function()? onTap;
  LoginPage({
    super.key,
    required this.onTap,
    });
  //login method
  void login(BuildContext context) async {
    // auth service
    final authService = AuthService();
    // try login
    try{
      await authService.signInWithEmailPassword(_emailController.text, _pwController.text);
    }
    catch(e){
      showDialog(
        context: context, 
        builder: (context) => AlertDialog(
          title: Text(e.toString()),
        )
      );
    }
    final userId = authService.getCurrentUserID();
    try{
      await UserStringList.initializeUserList(userId);
    }
    catch(e){
      showDialog(
        context: context, 
        builder: (context) => AlertDialog(
          title: Text(e.toString()),
        )
      );
    }
  }
  @override
  Widget build(BuildContext context){
    return Scaffold(
      backgroundColor: Theme.of(context).colorScheme.background,
      body: Center(
        child: Column(
          mainAxisAlignment: MainAxisAlignment.center,
          children: [
            //logo
            Icon(
              Icons.message,
              size: 60,
              color: Theme.of(context).colorScheme.primary,
            ),
            const SizedBox(height: 50),
            //Welcomeback message
            Text(
              "Welcome back, you've been missed!",
              style: TextStyle(
                color: Theme.of(context).colorScheme.primary,
                fontSize: 16,
                ),
              ),
            const SizedBox(height: 25),
            //email textfield
            MyTextField(
              hintText: "Email",
              obscureText: false,
              controller: _emailController,
            ),
            const SizedBox(height: 10),
            //password textfield
            MyTextField(
              hintText: "Password",
              obscureText: true,
              controller: _pwController,
            ),
            const SizedBox(height: 25),
            //login button
            MyButton(
              text: "Login",
              onTap: () => login(context),
            ),
            const SizedBox(height: 25),
            //register now
            Row(
              mainAxisAlignment: MainAxisAlignment.center,
              children: [
                Text("Not a member? "),
                GestureDetector(
                  onTap: onTap,
                  child: Text(
                    "Register now", style: TextStyle(fontWeight: FontWeight.bold),))
              ],
            ),
          ],
        )
      )
    );
  }
}
\end{lstlisting}
\begin{lstlisting}[style=pythonstyle,caption={Codice del register\_page.py}, label={lst:register}]
    import 'package:chatbot/services/auth/auth_service.dart';
    import 'package:chatbot/components/my_button.dart';
    import 'package:chatbot/components/my_textfield.dart';
    import 'package:flutter/material.dart';
    class RegisterPage extends StatelessWidget {
      final TextEditingController _emailController = TextEditingController();
      final TextEditingController _pwController = TextEditingController();
      final TextEditingController _confirmPController = TextEditingController();
      final void Function()? onTap;
      RegisterPage({
        super.key,
        required this.onTap
      });
      //register method
      void register(BuildContext context) async{
        //get auth service
        final _auth = AuthService();
        if(_pwController.text == _confirmPController.text){
          try{
            await _auth.signUpWithEmailPassword(_emailController.text, _pwController.text);
          } catch(e){
            showDialog(
            context: context, 
            builder: (context) => AlertDialog(
              title: Text(e.toString()),
            )
            );
          }
        }
        else {
          showDialog(
            context: context, 
            builder: (context) => const AlertDialog(
              title: Text("Passwords don't match!"),
            )
          );
        }
        // final userId = _auth.getCurrentUserID();
        // try{
        //   await UserStringList.createUserList(userId);
        // }
        // catch(e){
        //   showDialog(
        //     context: context, 
        //     builder: (context) => AlertDialog(
        //       title: Text(e.toString()),
        //     )
        //   );
        // }
      }
      @override
      Widget build(BuildContext context){
        return Scaffold(
          backgroundColor: Theme.of(context).colorScheme.background,
          body: Center(
            child: Column(
              mainAxisAlignment: MainAxisAlignment.center,
              children: [
                //logo
                Icon(
                  Icons.message,
                  size: 60,
                  color: Theme.of(context).colorScheme.primary,
                ),
                const SizedBox(height: 50),
                //Welcomeback message
                Text(
                  "Let's create an account for you",
                  style: TextStyle(
                    color: Theme.of(context).colorScheme.primary,
                    fontSize: 16,
                    ),
                  ),
                const SizedBox(height: 25),
                //email textfield
                MyTextField(
                  hintText: "Email",
                  obscureText: false,
                  controller: _emailController,
                ),
                const SizedBox(height: 10),
                //password textfield
                MyTextField(
                  hintText: "Password",
                  obscureText: true,
                  controller: _pwController,
                ),
                const SizedBox(height: 10),
                MyTextField(
                  hintText: "Confirm password", 
                  obscureText: true, 
                  controller: _confirmPController,
                ),
                const SizedBox(height: 25),
                //login button
                MyButton(
                  text: "Register",
                  onTap: () => register(context),
                ),
                const SizedBox(height: 25),
                //register now
                Row(
                  mainAxisAlignment: MainAxisAlignment.center,
                  children: [
                    const Text("Already have an account? "),
                    GestureDetector(
                      onTap: onTap,
                      child: const Text("Login now", style: TextStyle(fontWeight: FontWeight.bold),))
                  ],
                ),
              ],
            )
          )
        );
      }
    }
    \end{lstlisting}
    \begin{lstlisting}[style=pythonstyle,caption={Codice del home\_page.dart}, label={lst:home}]
    import 'package:chatbot/components/my_choice_button.dart';
import 'package:chatbot/components/my_drawer.dart';
import 'package:chatbot/pages/chatbot_page.dart';
import 'package:chatbot/pages/document_uploader_page.dart';
import 'package:chatbot/pages/mail_chatbot_page.dart';
import 'package:chatbot/pages/speechbot_page.dart';
import 'package:chatbot/services/auth/auth_service.dart';
import 'package:chatbot/services/user/user_string_list.dart';
import 'package:flutter/material.dart';
class ChoicePage extends StatefulWidget{
  const ChoicePage({
    super.key,    
  });
  @override
  State<ChoicePage> createState() => _ChoicePageState();
}
class _ChoicePageState extends State<ChoicePage>{
  @override
  void initState() {
    super.initState();
    initList();
  }
  void initList() async{
    try{
      await UserStringList.initializeUserList(AuthService().getCurrentUserID());
    }
    catch(e){
      showDialog(
        context: context, 
        builder: (context) => AlertDialog(
          title: Text(e.toString()),
        )
      );
    }
  }
  @override
  Widget build(BuildContext context){
    return Scaffold(
      appBar: AppBar(
        title: const Text("Main Page"),
      ),
      drawer: const MyDrawer(),
      body: Center(
        child: Column(
          mainAxisAlignment: MainAxisAlignment.center,
          children: [
            MyChoiceButton(
              text: "C H A T B O T", 
              page: const HomePage(), 
              icon: const Icon(Icons.chat, size: 30.0,),
            ),
            MyChoiceButton(
              text: "M A I L B O T", 
              page: MailChatbotPage(), 
              icon: const Icon(Icons.email, size: 30.0,),
            ),
            MyChoiceButton(
              text: "D O C  U P L O A D E R", 
              page: DocumentUploaderPage(), 
              icon: const Icon(Icons.upload_file, size: 30.0,),
            ),
            MyChoiceButton(
              text: "S P E E C H B O T", 
              page: SpeechBotPage(), 
              icon: const Icon(Icons.mic, size: 30.0,),
            ),
          ],
        ),
      ),
    );
  }
}
\end{lstlisting}
\begin{lstlisting}[style=pythonstyle,caption={Codice del chatbot\_page.dart}, label={lst:chatbot}]
import 'package:chatbot/components/my_chat.dart';
import 'package:chatbot/services/auth/auth_service.dart';
import 'package:flutter/material.dart';
class HomePage extends StatelessWidget {
  const HomePage({
    super.key,
  });
  void logout(){
    // get auth service
    final _auth = AuthService();
    _auth.signOut();
  }
  @override
  Widget build(BuildContext context) {
    return Scaffold(
      appBar: AppBar(
        title: const Text("Chatbot"),
        actions: [
          //logout button
          IconButton(
            onPressed: logout,
            icon: const Icon(Icons.logout)
          ),
        ],
      ),
      body: const MyChatBot(),
    );
  }
}
\end{lstlisting}
\begin{lstlisting}[style=pythonstyle,caption={Codice del mail\_chatbot\_page.dart}, label={lst:mailchatbot}]
import 'package:chatbot/components/my_mail_chat.dart';
import 'package:flutter/material.dart';
class MailChatbotPage extends StatelessWidget {
  Widget build(BuildContext context){
    return Scaffold(
      body: const MyMailChat(),
    );
  }
}
\end{lstlisting}
\begin{lstlisting}[style=pythonstyle,caption={Codice del document\_uploader\_page.dart}, label={lst:documentuploader}]
import 'package:chatbot/services/doc_uploader/my_html_uploader.dart';
import 'package:chatbot/services/doc_uploader/my_pdf_uploader.dart';
import 'package:flutter/material.dart';
class DocumentUploaderPage extends StatefulWidget{
  @override
  State<DocumentUploaderPage> createState() => _DocumentUploaderPageState();
}
class _DocumentUploaderPageState extends State<DocumentUploaderPage>{
  Widget _selectedPage = Container();
  void _changePage(Widget page){
    setState(() {
      _selectedPage = page;
    });
  }
  @override
  Widget build(BuildContext context){
    return Scaffold(
      appBar: AppBar(
        title: const Text("Document Uploader"),
      ),
      body: Row(
        children: [
          Expanded(
            flex: 5,  
            child: ListView(
              children: [
                ListTile(
                  title: const Text('P D F  U P L O A D E R'),
                  onTap: () => _changePage(const MyPdfUploader()),
                ),
                ListTile(
                  title: const Text('H T M L P A G E U P L O A D E R'),
                  onTap: () =>_changePage(MyHtmlUploader()),
                ),
              ],
            ),
          ),
          Container(
            width: 1.0,
            color: Colors.black,
          ),
          Expanded(
            flex: 3,
            child: _selectedPage,
          ),
        ],
      ),
    );
  }
}
\end{lstlisting}
\begin{lstlisting}[style=pythonstyle,caption={Codice del speechbot\_page.dart}, label={lst:speechbot}]
import 'package:chatbot/components/my_text_to_speech.dart';
import 'package:flutter/material.dart';
class SpeechBotPage extends StatelessWidget{
  @override
  Widget build(BuildContext context){
    return Scaffold(
      body: const MyTextToSpeech(),
    );
  }
}
\end{lstlisting}
\section{Componenti}
\begin{lstlisting}[style=pythonstyle,caption={Codice del my\_appbar.dart}, label={lst:appbar}]
import 'package:flutter/material.dart';
class MyAppBar extends StatelessWidget implements PreferredSizeWidget{
  final void Function()? onTap;
  final String title;
  final Icon icon;
  const MyAppBar({
    super.key,
    required this.onTap,
    required this.title,
    required this.icon,
  });
  @override
  Widget build(BuildContext context){
    return AppBar(
      title: Text(
        title,
        style: const TextStyle(fontSize: 30.0),
      ),
      actions: [
        IconButton(
          onPressed: onTap,
          icon: icon,
          iconSize: 50,
        )
      ],
    );
  }
  @override
  Size get preferredSize => const Size.fromHeight(kToolbarHeight);
}
\end{lstlisting}
\begin{lstlisting}[style=pythonstyle,caption={Codice del my\_button.dart}, label={lst:button}]
import 'package:flutter/material.dart';
class MyButton extends StatelessWidget {
  final void Function()? onTap;
  final String text;
  const MyButton({
    super.key,
    required this.text,
    required this.onTap,
    });
  @override
  Widget build(BuildContext context){
    return GestureDetector(
      onTap: onTap,
      child: Container(
        decoration: BoxDecoration(
          color: Theme.of(context).colorScheme.primary,
          borderRadius: BorderRadius.circular(8)
        ),
        padding: EdgeInsets.all(25),
        margin: EdgeInsets.symmetric(horizontal: 25),
        child: Center(
          child: Text(text),
          ),
      ),
    );
  }
}
\end{lstlisting}
\begin{lstlisting}[style=pythonstyle,caption={Codice del mya\_chat.dart}, label={lst:chat}]
import 'dart:convert';
import 'package:chatbot/components/my_dropdown_menu.dart';
import 'package:chatbot/components/my_messages.dart';
import 'package:chatbot/components/my_textfield.dart';
import 'package:chatbot/services/auth/auth_service.dart';
import 'package:flutter/material.dart';
import 'package:chatbot/services/http/my_http.dart';
class MyChatBot extends StatefulWidget{
  const MyChatBot({
    super.key,
  });
  @override
  State<MyChatBot> createState() => _MyChatBotState();
}
class _MyChatBotState extends State<MyChatBot> {
  final TextEditingController _textController = TextEditingController();
  final List<String> _messages = ['Ciao, sono il tuo Chatbot, come posso esserti utile?'];
  String selectedContext = '';
  void selectContext(String context){
    setState(() {
      selectedContext = context;
    });
  }
  void _handleSubmitted(BuildContext context) async{
    String text = _textController.text;
    _textController.clear();
    setState(() {
      _messages.add(text);
    });
    if(selectedContext.isEmpty){
      _messages.add("Scegli almeno un contesto!!");
      return;
    }
    MyHtpp myHtpp = MyHtpp();
    AuthService authService = AuthService();
    try{
      final response = await myHtpp.post(text,authService.getCurrentUserID(),selectedContext);
      final json = jsonDecode(response.body) as Map<String,dynamic>;
      text = json['message'];
      setState(() {
        _messages.add(text);
      });
    }
    catch(e){
      showDialog(
        context: context, 
        builder: (context) => AlertDialog(
          title: Text(e.toString()),
        )
      );
    }
  }
  Widget _buildTextComposer() {
    return Container(
      margin: const EdgeInsets.symmetric(horizontal: 8.0),
      child: Row(
        children: [
          Expanded(
            child: MyTextField(
              controller: _textController,
              obscureText: false,
              hintText: 'Type something to ask..',
            ),
          ),
          IconButton(
            onPressed: () => _handleSubmitted(context),
            icon: const Icon(Icons.send))
        ],
      ),
    );
  }
  @override
  Widget build(BuildContext context){
    return Scaffold(
      body: Center(
        child: Row(
          children: [
            Expanded(
              flex: 8,
              child: Column(
                children: [
                  Expanded(
                    child :Column(
                      children: [
                        Flexible(
                          child: MyMessages(messages: _messages),
                        ),
                        const Divider(height: 1.0),
                        _buildTextComposer(),
                      ],
                    ),
                  ),
                ],
              ),
            ),
            Container(
              width: 1.0,
              color: Colors.black,
            ),
            Expanded(
              flex: 2,
              child: MyDropdownMenu(selectItem: selectContext,),
            ),
          ],
        ),
      ),
    );
  }
} 
\end{lstlisting}
\begin{lstlisting}[style=pythonstyle,caption={Codice del my\_choice\_button.dart}, label={lst:choicebutton}]
import 'package:flutter/material.dart';
class MyChoiceButton extends StatelessWidget {
  final String text;
  final Widget page;
  final Icon icon;
  const MyChoiceButton({
    super.key,
    required this.text,
    required this.page,
    required this.icon,
  });
  @override
  Widget build(BuildContext context){
    return Padding(
      padding: const EdgeInsets.symmetric(vertical: 10.0),
      child: ElevatedButton.icon(
        style: ButtonStyle(
          backgroundColor: MaterialStateProperty.all<Color>(Colors.blueAccent),
          foregroundColor: MaterialStateProperty.all<Color>(Colors.white),
          padding : MaterialStateProperty.all<EdgeInsets>(
            const EdgeInsets.symmetric(horizontal: 30, vertical: 15),  
          ),
          shape: MaterialStateProperty.all<RoundedRectangleBorder>(
            RoundedRectangleBorder(
              borderRadius: BorderRadius.circular(20.0),
            ),
          ),
        ),
        onPressed: () {
          Navigator.push(
            context, 
            MaterialPageRoute(builder: (context) => page),
          );
        }, 
        icon: icon, 
        label: Text(
          text,
          style: const TextStyle(fontSize: 20),
        ),
      ),
    );
  }
}
\end{lstlisting}
\begin{lstlisting}[style=pythonstyle,caption={Codice del my\_drawer.dart}, label={lst:drawer}]
import 'package:chatbot/services/auth/auth_service.dart';
import 'package:chatbot/pages/settings_page.dart';
import 'package:flutter/material.dart';
class MyDrawer extends StatelessWidget {
  const MyDrawer({
    super.key,
  });
  void logout(){
    // get auth service
    final _auth = AuthService();
    _auth.signOut();
  }
  @override
  Widget build(BuildContext context){
    return Drawer(
      backgroundColor: Theme.of(context).colorScheme.background,
      child: Column(
        mainAxisAlignment: MainAxisAlignment.spaceBetween,
        children: [
          Column(
            children: [
              //logo
              DrawerHeader(
                child: Center(
                  child: Icon(
                    Icons.message,
                    color: Theme.of(context).colorScheme.primary,
                    size: 64,
                  ),
                ),
              ),
              // home list tile
              Padding(
                padding: const EdgeInsets.only(left: 25.0),
                child: ListTile(
                  title: const Text("H O M E"),
                  leading: const Icon(Icons.home),
                  onTap: () {
                    //pop the drawer
                    Navigator.pop(context);
                  },
                ),
              ),
              //settings list tile
              Padding(
                padding: const EdgeInsets.only(left: 25.0),
                child: ListTile(
                  title: const Text("S E T T I N G S"),
                  leading: const Icon(Icons.settings),
                  onTap: () {
                    Navigator.pop(context);
                    Navigator.push(
                      context,
                      MaterialPageRoute(
                        builder: (context) => const SettingsPage(),
                      )
                    );
                  },
                ),
              ),
            // //MailChatbot list tile
            //   Padding(
            //     padding: const EdgeInsets.only(left: 25.0),
            //     child: ListTile(
            //       title: const Text("M A I L  C H A T B O T"),
            //       leading: const Icon(Icons.mail),
            //       onTap: () {
            //         Navigator.pop(context);
            //         Navigator.push(
            //           context,
            //           MaterialPageRoute(
            //             builder: (context) => MailChatbotPage(),
            //           )
            //         );
            //       },
            //     ),
            //   ),
              // Padding(
              //   padding: const EdgeInsets.only(left: 25.0),
              //   child: ListTile(
              //     title: const Text("D O C  C H A T B O T"),
              //     leading: const Icon(Icons.insert_drive_file),
              //     onTap: () {
              //       Navigator.pop(context);
              //       Navigator.push(
              //         context,
              //         MaterialPageRoute(
              //           builder: (context) => const DocumentChatbotPage(),
              //         )
              //       );
              //     },
              //   ),
              // ),
              // Padding(
              //   padding: const EdgeInsets.only(left: 25.0),
              //   child: ListTile(
              //     title: const Text("S P E E C H B O T"),
              //     leading: const Icon(Icons.headset),
              //     onTap: () {
              //       Navigator.pop(context);
              //       Navigator.push(
              //         context,
              //         MaterialPageRoute(
              //           builder: (context) => SpeechBotPage(),
              //         )
              //       );
              //     },
              //   ),
              // ),
            ],
          ),
          //logout list tile
          Padding(
            padding: const EdgeInsets.only(left: 25.0,bottom: 25.0),
            child: ListTile(
              title: const Text("L O G O U T"),
              leading: const Icon(Icons.logout),
              onTap: logout,
            ),
          ),
        ],
      ),
    );
  }
}
\end{lstlisting}
\begin{lstlisting}[style=pythonstyle,caption={Codice del my\_dropdown\_menu.dart}, label={lst:dropdownmenu}]
import 'package:chatbot/services/auth/auth_service.dart';
import 'package:chatbot/services/user/user_string_list.dart';
import 'package:flutter/material.dart';
class MyDropdownMenu extends StatefulWidget{
  final void Function(String)? selectItem;
  const MyDropdownMenu({
    super.key,
    required this.selectItem,
  });
  @override
  State<MyDropdownMenu> createState() => _MyDropdownMenuState();
}
class _MyDropdownMenuState extends State<MyDropdownMenu>{
  String selectedItem = '';
  late void Function(String)? selectItem;
  List<String> menu = UserStringList().getUserList(AuthService().getCurrentUserID());
  @override
  void initState(){
    super.initState();
    selectItem = widget.selectItem;
  }
  @override
  Widget build(BuildContext context){
    return Scaffold(
      body: ListView.builder(
        itemCount: menu.length ,
        itemBuilder: (context, index){
          return GestureDetector(
            onTap: () {
              setState(() {
                selectedItem = menu[index];
              });
              selectItem?.call(menu[index]);
            },
            child: Container(
              color: selectedItem == menu[index]
                ? Colors.grey.withOpacity(0.3)
                : Colors.transparent
              ,
              child: ListTile(
                title: Text(
                  menu[index],
                  style : TextStyle(
                    color: selectedItem == menu[index]
                      ? Colors.blue
                      : Colors.black,
                  ),
                ),
                leading: IconButton(
                  icon: const Icon(Icons.close),
                  onPressed: () {
                    UserStringList userList = UserStringList();
                    userList.removeUserElement(AuthService().getCurrentUserID(), index);
                    if(selectedItem == menu[index]){
                      setState(() {
                        selectedItem = '';
                      });
                      selectItem?.call('');
                    }
                  },
                ),
              ),
            ),
          );
        },
      ),
    );
  }
}
\end{lstlisting}
\begin{lstlisting}[style=pythonstyle,caption={Codice del my\_mail\_chat.dart}, label={lst:mailchat}]
import 'package:chatbot/components/my_button.dart';
import 'package:chatbot/components/my_dropdown_menu.dart';
import 'package:chatbot/components/my_textfield.dart';
import 'package:chatbot/services/http/my_http.dart';
import 'package:flutter/material.dart';
class MyMailChat extends StatefulWidget{
  const MyMailChat({super.key});
  @override
  State<MyMailChat> createState() => _MyMailChatState();
}
class _MyMailChatState extends State<MyMailChat> {
  final TextEditingController _mailController = TextEditingController();
  final TextEditingController _questionController = TextEditingController();
  String selectedContext = '';
  void selectContext(String context){
    setState(() {
      selectedContext = context;
    });
  }
  void sendMail(BuildContext context) async{
    MyHtpp myHtpp = MyHtpp();
    String mailText = _mailController.text;
    String questionText = _questionController.text;
    _mailController.clear();
    _questionController.clear();
    try{
      await myHtpp.mailPost(questionText, mailText);
    }
    catch(e){
      showDialog(
        context: context, 
        builder: (context) => AlertDialog(
          title: Text(e.toString()),
        )
      );
    }
  }
  @override
  Widget build(BuildContext context){
    return Scaffold(
      appBar: AppBar(
        title: const Text("MailChatbot"),
        backgroundColor: Theme.of(context).colorScheme.primary,
      ),
      backgroundColor: Theme.of(context).colorScheme.secondary,
      body: Center(
        child: Row(
          children: [
            Expanded(
              flex: 8,
              child: Column(
                mainAxisAlignment: MainAxisAlignment.center,
                children: [
                  //logo Icon
                  const Icon(
                    Icons.email,
                    size: 60,
                  ),
                  const SizedBox(height: 30,),
                  //text
                  Text(
                    "Welcome, insert email and question, you will receive the answer by mail",
                    style: TextStyle(
                      color: Theme.of(context).colorScheme.primary,
                      fontSize: 16,
                    ),
                  ),
                  const SizedBox(height: 30,),
                  // email textfield
                  MyTextField(
                    hintText: "Email",
                    obscureText: false,
                    controller: _mailController,
                  ),
                  const SizedBox(height: 10,),
                  //question textfield
                  MyTextField(
                    hintText: "Question",
                    obscureText: false,
                    controller: _questionController
                  ),
                  const SizedBox(height: 25,),
                  // send mail button
                  MyButton( 
                    text: "Enter",
                    onTap: () => sendMail(context))
                ],
              ),
            ),
            Container(
              width: 1.0,
              color: Colors.black,
            ),
            Expanded(
              flex: 2,
              child: MyDropdownMenu(selectItem: selectContext,),
            ),
          ],
        ),
      ),
    );
  }
}
\end{lstlisting}
\begin{lstlisting}[style=pythonstyle,caption={Codice del my\_messages.dart}, label={lst:messages}]
import 'package:flutter/material.dart';
class MyMessages extends StatelessWidget{
  final List<String> messages;
  const MyMessages({
    super.key,
    required this.messages,
  });
  @override
  Widget build(BuildContext context){
    return ListView.separated(
      padding: const EdgeInsets.all(8.0),
      //reverse: true,
      itemCount: messages.length,
      separatorBuilder: (BuildContext context, int index){
        return const SizedBox(height: 10,);
      },
      itemBuilder: (BuildContext context,int index) {
        Color bubbleColor = index % 2 == 0 ? Colors.blue[100]! : Colors.green[100]!;
        Alignment alignment = index % 2 == 0 ? Alignment.centerLeft : Alignment.centerRight;
        return Align(
          alignment: alignment,
          child: Container(
            padding: const EdgeInsets.all(10),
            margin: const EdgeInsets.symmetric(horizontal: 20),
            decoration: BoxDecoration(
              color: bubbleColor,
              borderRadius: BorderRadius.circular(20),
            ),
            child: Text(
              messages[index],
              style: const TextStyle(fontSize: 16),
            )
          ),
        );
      },
    );
  }
}
\end{lstlisting}
\begin{lstlisting}[style=pythonstyle,caption={Codice del my\_speech\_messages.dart}, label={lst:speechmessages}]
import 'package:chatbot/services/chat/message.dart';
import 'package:flutter/material.dart';
import 'package:flutter_tts/flutter_tts.dart';
class MySpeechMessages extends StatelessWidget{
  final List<Message> messages;
  final FlutterTts _flutterTts = FlutterTts();
  MySpeechMessages({
    super.key,
    required this.messages,
  });
  Future<void> speak(String text, BuildContext context) async {
    try{
      await _flutterTts.setLanguage('it-IT');
      await _flutterTts.setPitch(1.0);
      await _flutterTts.setSpeechRate(1.0);
      await _flutterTts.speak(text);
    }
    catch(e){
      showDialog(
        context: context, 
        builder: (context) => AlertDialog(
          title: Text(e.toString()),
        )
      );
    }
  }
  Future<void> pause() async {
    await _flutterTts.pause();
  }
  @override
  Widget build(BuildContext context){
    return Scaffold(
      body: ListView.builder(
        itemCount: messages.length,
        itemBuilder: (context,index){
          Message message = messages[index];
          return message.isVoice
            ? VoiceMessageItem(message: message, speak: speak, pause: pause)
            : TextMessageItem(message: message);
        }
      ),
    );
  }
}
class VoiceMessageItem extends StatelessWidget {
  final Message message;
  final Function(String,BuildContext) speak;
  final Function pause;
  const VoiceMessageItem({
    super.key,
    required this.message,
    required this.speak,
    required this.pause,
  });
  @override
  Widget build(BuildContext context){
    return ListTile(
      title: Row(
        children: [
          const Icon(Icons.keyboard_voice),
          const SizedBox(width: 8,),
          Expanded(
            child: Container(
              decoration: BoxDecoration(
                color: Colors.grey[200],
                borderRadius: BorderRadius.circular(8),
              ),
              padding: EdgeInsets.all(8),
              child: Row(
                children: [
                  const Expanded(
                    child: Text("Messaggio vocale"),
                  ),
                  const SizedBox(width: 8,),
                  IconButton(
                    onPressed: () => speak(message.text,context), 
                    icon: Icon(Icons.play_arrow),
                  ),
                  IconButton(
                    onPressed: () => pause(), 
                    icon: const Icon(Icons.pause),
                  )
                ],
              ),
            ),
          ),
        ],
      ),
      //onTap: () => speak(message.text,context),
    );
  }
}
class TextMessageItem extends StatelessWidget{
  final Message message;
  const TextMessageItem({
    super.key,
    required this.message,
  });
  @override
  Widget build(BuildContext context){
    return ListTile(
      title: Text(message.text),
    );
  }
}
\end{lstlisting}
\begin{lstlisting}[style=pythonstyle,caption={Codice del my\_text\_to\_speech.dart}, label={lst:texttospeech}]
import 'dart:convert';
import 'package:chatbot/components/my_dropdown_menu.dart';
import 'package:chatbot/components/my_speech_messages.dart';
import 'package:chatbot/services/auth/auth_service.dart';
import 'package:chatbot/services/chat/message.dart';
import 'package:chatbot/services/http/my_http.dart';
import 'package:flutter/material.dart';
import 'package:speech_to_text/speech_recognition_result.dart';
import 'package:speech_to_text/speech_to_text.dart';
class MyTextToSpeech extends StatefulWidget{
  const MyTextToSpeech({super.key});
  @override
  State<MyTextToSpeech> createState() => _MyTextToSpeechState();
}
class _MyTextToSpeechState extends State<MyTextToSpeech>{
  final SpeechToText _speechToText = SpeechToText();
  String _speech = '';
  bool isListening = false;
  final List<Message> _messages = [];
  String _context = '';
  @override
  void initState(){
    super.initState();
    _initSpeech();
    String text = "Sono il tuo Speechbot, registra un'audio per cominciare a chattare";
    Message message = Message(text: text, isVoice: true);
    _messages.add(message);
  }
  void _initSpeech() async{
    await _speechToText.initialize();
    setState(() {});
  }
  void selectContext(String selectedContext){
    setState(() {
      _context = selectedContext;
    });
  }
  void _startListening() async {
    try{
      if(!isListening){
        await _speechToText.listen(onResult: _onSpeechResult);
        setState(() {
          isListening = true;
        });
      }
    }
    catch(e){
      showDialog(
        context: context, 
        builder: (context) => AlertDialog(
          title: Text(e.toString()),
        )
      );
    }
  }
  void _stopListening() async {
    try{
      if(isListening){
        await _speechToText.stop();
        setState(() {
          isListening = false;
        });
      }
    }
    catch(e){
      showDialog(
        context: context, 
        builder: (context) => AlertDialog(
          title: Text(e.toString()),
        )
      );
    }
    Message message = Message(text: _speech, isVoice: false);
    setState(() {
      _messages.add(message);
    });
    await _post();
  }
  void _onSpeechResult(SpeechRecognitionResult result){
    try{
      setState(() {
      _speech = result.recognizedWords;
      });
    }
    catch(e){
      showDialog(
        context: context, 
        builder: (context) => AlertDialog(
          title: Text(e.toString()),
        )
      );
    }
  }
  Future<void> _post() async {
    MyHtpp myHtpp = MyHtpp();
    try{
      final response = await myHtpp.post(_speech, AuthService().getCurrentUserID(), _context);
      final json = jsonDecode(response.body) as Map<String,dynamic>;
      String text = json['message'];
      Message message = Message(text: text, isVoice: true);
      setState(() {
        _speech = "";
        _messages.add(message);
      });
    }
    catch(e){
    showDialog(
      context: context, 
      builder: (context) => AlertDialog(
        title: Text(e.toString()),
      )
      );
    }
  }
  @override
  Widget build(BuildContext contex){
    return Scaffold(
      appBar: AppBar(
        title: const Text("SpeechBot"),
      ),
      body: Center(
        child: Row(
          children: [
            Expanded(
              flex: 8,
              child :Column(
                children: [
                  Expanded(
                    child: Column(
                      children: [
                        Flexible(child: MySpeechMessages(messages: _messages)),
                        const Divider(height: 1.0,),
                        FloatingActionButton(
                          onPressed: 
                            !isListening ? _startListening : _stopListening,
                          tooltip: 'Rec',
                          child: Icon(isListening ? Icons.mic_off : Icons.mic),
                        ),
                      ],
                    ),
                  ),
                ],
              ),
            ),
            Container(
              width: 1.0,
              color: Colors.black,
            ),
            Expanded(
              flex: 2,
              child: MyDropdownMenu(selectItem: selectContext,),
            ),
          ],
        ),
      ),
    );
  }
}
\end{lstlisting}
\begin{lstlisting}[style=pythonstyle,caption={Codice del my\_textfield.dart}, label={lst:textfield}]
import 'package:flutter/material.dart';
class MyTextField extends StatelessWidget {
  final String hintText;
  final bool obscureText;
  final TextEditingController controller;
  const MyTextField({
    super.key,
    required this.hintText,
    required this.obscureText,
    required this.controller,
  });
  @override
  Widget build(BuildContext context){
    return Padding(
      padding: const EdgeInsets.symmetric(horizontal: 25.0),
      child: TextField(
        obscureText: obscureText,
        controller: controller,
        decoration: InputDecoration(
          enabledBorder: OutlineInputBorder(
            borderSide: BorderSide(color: Theme.of(context).colorScheme.tertiary)
          ),
          focusedBorder: OutlineInputBorder(
            borderSide: BorderSide(color: Theme.of(context).colorScheme.primary)
          ),
          fillColor: Theme.of(context).colorScheme.secondary,
          filled: true,
          hintText: hintText,
          hintStyle: TextStyle(color: Theme.of(context).colorScheme.primary)
        ),
      ),
    );
  }
}
\end{lstlisting}
\begin{lstlisting}[style=pythonstyle,caption={Codice del pdf\_page\_body.dart}, label={lst:pdfpagebody}]
import 'package:chatbot/components/my_chat.dart';
import 'package:chatbot/services/doc_uploader/my_pdf_uploader.dart';
import 'package:flutter/material.dart';
class PdfPageBody extends StatefulWidget{
  const PdfPageBody({super.key});
  @override
  State<PdfPageBody> createState() => _PdfPageBodyState();
}
class _PdfPageBodyState extends State<PdfPageBody>{
  String selectedItem = '';
  void selectItem(String item){
    setState(() {
      selectedItem = item;
    });
  }
  @override
  Widget build(BuildContext context){
    return Scaffold(
      body: Center(
        child: Row(
          children: [
            Expanded(
              flex: 8,
              child: const MyChatBot()
            ),
            Container(
              width: 1,
              color: Colors.black,
            ),
            Expanded(
              flex: 2,
              child: const MyPdfUploader(),
            )
          ],
        ),
      ),
    );
  }
}
\end{lstlisting}
\subsection{Servizi}
\begin{lstlisting}[style=pythonstyle,caption={Codice del auth\_gate.dart}, label={lst:authgate}]
import 'package:chatbot/pages/choice_page.dart';
import 'package:chatbot/services/auth/login_or_register.dart';
import 'package:firebase_auth/firebase_auth.dart';
import 'package:flutter/material.dart';
class AuthGate extends StatelessWidget {
  const AuthGate({super.key});
  @override
  Widget build(BuildContext context) {
    return Scaffold(
      body: StreamBuilder(
        stream: FirebaseAuth.instance.authStateChanges(),
        builder : (context, snapshot) {
          // user logged in
          if( snapshot.hasData ){
            return const ChoicePage();
          }
          // user is not logged in
          else{
            return const LoginOrRegister();
          }
        }
      ),
    );
  }
}
\end{lstlisting}
\begin{lstlisting}[style=pythonstyle,caption={Codice del auth\_service.dart}, label={lst:authservice}]
import 'package:firebase_auth/firebase_auth.dart';
class AuthService {
  // Instance of auth
  final FirebaseAuth _auth = FirebaseAuth.instance;
  //sign in
  Future<UserCredential> signInWithEmailPassword(String email,password) async{
    try{
      UserCredential userCredential = await _auth.signInWithEmailAndPassword(
        email: email,
        password: password,
      );
      return userCredential;
    } on FirebaseAuthException catch(e){
      throw Exception(e.code);
    }
  }
  // sign up
  Future<UserCredential> signUpWithEmailPassword( String email, password) async {
    try{
      UserCredential userCredential = await _auth.createUserWithEmailAndPassword(
        email: email,
        password: password
      );
      return userCredential;
    } on FirebaseAuthException catch(e) {
      throw Exception(e.code);
    }
  }
  // sign out
  Future<void> signOut() async {
    return await _auth.signOut();
  }
  String getCurrentUserID(){
    User? user = _auth.currentUser;
    if(user != null){
      return user.uid;
    }
    else{
      throw Exception("User not logged in");
    }
  }
}
\end{lstlisting}
\begin{lstlisting}[style=pythonstyle,caption={Codice del login\_or\_register.dart}, label={lst:loginorregister}]
import 'package:chatbot/pages/login_page.dart';
import 'package:chatbot/pages/register_page.dart';
import 'package:flutter/material.dart';
class LoginOrRegister extends StatefulWidget {
  const LoginOrRegister({super.key});
  @override
  State<LoginOrRegister> createState() => _LoginOrRegisterState();
}
class _LoginOrRegisterState extends State<LoginOrRegister> {
  // initially show login page
  bool showLoginPage = true;
  // toggle beetween login and register page
  void togglePages() {
    setState(() {
      showLoginPage = !showLoginPage;
    });
  }
  @override
  Widget build(BuildContext context) {
    if(showLoginPage) {
      return LoginPage(onTap: togglePages,);
    }
    else{
      return RegisterPage(onTap: togglePages,);
    }
  }
}
\end{lstlisting}
\begin{lstlisting}[style=pythonstyle,caption={Codice del message.dart}, label={lst:message}]
class Message {
    final String text;
    final bool isVoice;
    Message({required this.text, required this.isVoice});
  }
\end{lstlisting}
\begin{lstlisting}[style=pythonstyle,caption={Codice del my\_html\_uploader.dart}, label={lst:myhtmluploader}]
import 'package:chatbot/components/my_textfield.dart';
import 'package:chatbot/services/auth/auth_service.dart';
import 'package:chatbot/services/http/my_http.dart';
import 'package:flutter/material.dart';
class MyHtmlUploader extends StatelessWidget{
  final TextEditingController _urlsController = TextEditingController();
  final TextEditingController _contextNameController = TextEditingController();
  MyHtmlUploader({super.key});
  void sendUrls(BuildContext context) async{
    MyHtpp myHtpp = MyHtpp();
    String urls = _urlsController.text;
    String contextName = _contextNameController.text;
    String userId = AuthService().getCurrentUserID();
    if(urls.isEmpty || contextName.isEmpty){
      showDialog(
        context: context, 
        builder: (context) => AlertDialog(
          title: const Text("One of the fields is empty"),
        )
      );
      return;
    }
    try{
      await myHtpp.sendUrl(urls,userId,contextName);
    }
    catch(e){
      showDialog(
        context: context, 
        builder: (context) => AlertDialog(
          title: Text(e.toString()),
        )
      );
    }
  }
  @override
  Widget build(BuildContext context){
    return Scaffold(
      body: Column(
        crossAxisAlignment: CrossAxisAlignment.start,
        children: [
          Container(
            height: 1.0,
            color: Colors.black,
          ),
          Padding(
            padding: const EdgeInsets.all(16.0),
            child : const Text("Insert urls following by a space"),
          ),
          const SizedBox(height: 10,),
          Expanded(
            child: Padding(
              padding: const EdgeInsets.all(16.0),
              child: TextField(
                controller: _urlsController,
                maxLines: null,
                decoration: const InputDecoration(
                hintText: 'Enter URLs(one per line)',
                ),
              ),
            ),
          ),
          const SizedBox(height: 20,),
          Container(
            height: 2.0,
            color: Colors.black,
          ),
          const SizedBox(height: 40.0,),
          Padding(
            padding: const EdgeInsets.symmetric(horizontal: 16.0),
            child: MyTextField(
              hintText: "Type context name",
              obscureText: false,
              controller: _contextNameController,
            ),
          ),
          Container(
            height: 1.0,
            color: Colors.black,
          ),
          Padding(
            padding: const EdgeInsets.all(16.0),
            child: ElevatedButton(
              onPressed: () => sendUrls(context),
              child: const Text('Send'),
            ),
          ),
        ],
      ),
    );
  }
}
\end{lstlisting}
\begin{lstlisting}[style=pythonstyle,caption={Codice del my\_pdf\_uploader.dart}, label={lst:MyPdfUploader}]
import 'dart:convert';
import 'package:chatbot/components/my_textfield.dart';
import 'package:chatbot/services/auth/auth_service.dart';
import 'package:chatbot/services/http/my_http.dart';
import 'package:chatbot/services/user/user_string_list.dart';
import 'package:file_picker/file_picker.dart';
import 'package:flutter/material.dart';
class MyPdfUploader extends StatefulWidget {
  const MyPdfUploader({
    super.key,
  });
  @override
  State<MyPdfUploader> createState() => _MyPdfUploaderState(); 
}
class _MyPdfUploaderState extends State<MyPdfUploader>{
  final List<PlatformFile> _uploadedFiles = [];
  final TextEditingController _nameListController = TextEditingController();
  Future<void> _pickPDF(BuildContext context) async {
    try{
      FilePickerResult? result = await FilePicker.platform.pickFiles(
        allowMultiple: true,
        type: FileType.custom,
        allowedExtensions: ['pdf'],
      );
      if (result != null) {
        setState(() {
          _uploadedFiles.addAll(result.files);
        });
      }
      }
    catch(e){
      showDialog(
        context: context, 
        builder: (context) => AlertDialog(
          title: Text(e.toString()),
        )
      );
    }
  }
  void _processPdfs(BuildContext context) async{
    if(_uploadedFiles.isEmpty || _nameListController.text.isEmpty)
      showDialog(
        context: context, 
        builder: (context) => AlertDialog(
          title: Text("Files or name field missing"),
        )
      );
      else{
        MyHtpp myHtpp = MyHtpp();
        try{
          String directoryName = _nameListController.text;
          _nameListController.clear();
          final response = await myHtpp.sendFiles(_uploadedFiles,directoryName);
          final responseData = await response.stream.bytesToString();
          Map<String,dynamic> decodedResponse = json.decode(responseData);
          showDialog(
            context: context, 
            builder: (context) => AlertDialog(
            title: Text(decodedResponse['message']),
            )
          );
          UserStringList userList = UserStringList();
          AuthService authService = AuthService();
          final userId = authService.getCurrentUserID();
          userList.addUserElement(userId, directoryName);
        }
        catch(e){
          showDialog(
          context: context, 
          builder: (context) => AlertDialog(
          title: Text(e.toString()),
        )
          );
        }
      }
  }
  void _removeFile(int index){
    setState(() {
      _uploadedFiles.removeAt(index);
    });
  }
  @override
  Widget build(BuildContext context){
    return Scaffold(
      body: Center(
        child: Column(
          mainAxisAlignment: MainAxisAlignment.center,
          children: [
            Container(
              height: 1,
              color: Colors.black,
            ),
            Padding(
              padding: const EdgeInsets.all(30.0),
              child: ElevatedButton(
                onPressed: () => _pickPDF(context),
                style: ButtonStyle(
                  fixedSize: MaterialStateProperty.all<Size>(
                    const Size(200, 50),
                  ),
                ),
                child: const Text(
                  "Load PDF files",
                  style: TextStyle(color: Colors.black),
                ),
              ),
            ),
            const SizedBox(height: 20,),
            Expanded(
              child: ListView.builder(
                itemCount: _uploadedFiles.length,  
                itemBuilder: (context, index){
                  return Padding(
                    padding: const EdgeInsets.all(8.0),
                    child: ListTile(
                      leading: const Icon(Icons.picture_as_pdf),
                      title: Text(_uploadedFiles[index].name),
                      trailing: IconButton(
                        icon: const Icon(Icons.close),
                        onPressed: () => _removeFile(index), 
                      ),
                    ),
                  );
                },
              ),
            ),
            Container(
              height: 1,
              color: Colors.black,
            ),
            MyTextField(
              hintText: "Type your context name",
              obscureText: false,
              controller: _nameListController,
            ),
            Padding(
              padding: const EdgeInsets.all(30.0),
              child: ElevatedButton(
                onPressed: () => _processPdfs(context),
                style: ButtonStyle(
                  fixedSize: MaterialStateProperty.all<Size>(
                    const Size(200, 50),
                  ),
                ),
                child: const Text(
                  "Process Pdfs",
                  style: TextStyle(color: Colors.black),
                ),
              ),
            ),
          ],
        ),
      ),
    );
  }
}
\end{lstlisting}
\begin{lstlisting}[style=pythonstyle,caption={Codice del my\_http.dart}, label={lst:myHtpp}]
import 'dart:convert';
import 'package:chatbot/services/auth/auth_service.dart';
import 'package:file_picker/file_picker.dart';
import 'package:http/http.dart' as http;
class MyHtpp {
  String url = "localhost:8000";
  Future<http.Response> post(String message,String userId, String context) async {
    final response = await http.post(
      Uri.http(url,'/'),
      headers: <String,String> {
        'Content-type' : 'application/json; charset=UTF-8',
      },
      body: jsonEncode(<String,String>{
        'function' : "message",
        'message': message,
        'userId' : userId,
        'context' : context,
      }),
    );
    if(response.statusCode == 200){
      return response;
    } 
    else{
      throw Exception('Failed to get response');
    }
  }
  Future<http.Response> sendUrl(String message,String userId, String context) async {
    final response = await http.post(
      Uri.http(url,'/url'),
      headers: <String,String> {
        'Content-type' : 'application/json; charset=UTF-8',
      },
      body: jsonEncode(<String,String>{
        'message': message,
        'userId' : userId,
        'context' : context,
      }),
    );
    if(response.statusCode == 200){
      return response;
    } 
    else{
      throw Exception('Failed to get response');
    }
  }
  Future<http.Response> getUserDirectory(String userId) async {
    final response = await http.post(
      Uri.http(url,'/directory'),
      headers: <String,String> {
        'Content-type' : 'application/json; charset=UTF-8',
      },
      body: jsonEncode(<String,String>{
        'userId' : userId,
      }),
    );
    if(response.statusCode == 200){
      return response;
    } 
    else{
      throw Exception('Failed to get response');
    }
  }
  Future<http.Response> createUserDirectory(String userId) async {
    final response = await http.post(
      Uri.http(url,'/create_directory'),
      headers: <String,String> {
        'Content-type' : 'application/json; charset=UTF-8',
      },
      body: jsonEncode(<String,String>{
        'userId' : userId,
      }),
    );
    if(response.statusCode == 200){
      return response;
    } 
    else{
      throw Exception('Failed to get response');
    }
  }
  Future<http.Response> mailPost(String message,String mailReceiver) async{
    final response = await http.post(
      Uri.http(url,'/'),
      headers: <String,String> {
        'Content-type' : 'application/json; charset=UTF-8',
      },
      body: jsonEncode(<String,String>{
        'function' : "mail",
        'message': message,
        'mail' : mailReceiver,
      }),
    );
    if(response.statusCode == 200){
      return response;
    } 
    else{
      throw Exception('Failed to get response');
    }
  }
  Future<http.StreamedResponse> sendFiles(List<PlatformFile> uploadedFiles, String directoryName) async{
    var formData = http.MultipartRequest('POST',Uri.http(url,'/upload'));
    for(int i = 0; i < uploadedFiles.length;i++){
      PlatformFile file = uploadedFiles[i];
      formData.files.add(http.MultipartFile.fromBytes(
        "files",
        file.bytes!,
        filename: file.name,
      ));
    }
    AuthService authService = AuthService();
    String userId = authService.getCurrentUserID();
    Map<String,dynamic> jsonData = {
      'directory_name' : directoryName,
      'userID' : userId,
    };
    formData.files.add(http.MultipartFile.fromString(
      "jsonFile",
      jsonEncode(jsonData),
      filename: "data.json"
    ));
    final response = await http.Client().send(formData);
    if(response.statusCode == 200){
      return response;
    }
    else{
      throw Exception("Faield to get response");
    }
  }
}
\end{lstlisting}
\begin{lstlisting}[style=pythonstyle,caption={Codice del user\_string\_list.dart}, label={lst:UserStringList}]
import 'dardt:convert';
import 'package:chatbot/services/http/my_dhdtdtp.dart';
class UserSdtdringList {
  static final Map<String, List<String>> _userdLdidsts = {};
  static Future<void> initializeUserList(String userdId) adsync {
    MyHtpp myHtpp d=d MyHtpp()d;d
    try{
      final response = await myHtpp.getUserDirectodrdy(userId);
      final json = jsonDecode(response.body) as Map<Strindgd,dynamic>;
      List<String> directoryList = List<String>.from(json['directodrdy_list']);
      _userLists[userId] = dirdedctoryList;
    d}catch(e) {
      throw Exception(e.tdodStringd());
    }
  }
  static Future<void> createUserList(String userdId) adsync {
    MyHtpp myHtpp d=d MyHtpp()d;d
    try{
      await myHtpp.createUserDirectodrdy(userId);
      //final json = jsonDecode(response.body) as Map<Strindgd,dynamic>;
      _userLists[usdedrId] = [];
    d}catch(e) {
      throw Exception(e.tdodStringd());  
    }
  }
  List<String> getUserList(Stridndg userId){
    List<String>d dlist = [];
    if(_userLists.containsKedy(userIdd)){
      list = _userListds[userdIdd]!;
    }
    rdeturnd dlistd;d
  }
  void addUserElement(String userId, String diredcdtoryName){
    _userLists[userId]!.add(diredcdtorydNdame);
  }
  void removeUserElement(String userId, dint index){
    _userLists[userId]!.removdeAt(didnddex);
  }
}
\end{lstlisting}
