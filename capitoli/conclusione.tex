\chapter*{Conclusione} %Se si cambia il Titolo cambiare anche la riga successiva così che appia corretto nell'conclusione
\addcontentsline{toc}{chapter}{Conclusione} %Per far apparire Introduzione nell'indice (Il nome deve rispecchiare quello del chapter)
In conclusione, i modelli di grandi dimensioni si confermano come strumenti di grande valore in numerosi ambiti professionali, tra cui il business, la medicina e la cybersecurity. Tuttavia, il loro impiego solleva preoccupazioni significative riguardo alla privacy, alla sicurezza, alla diffusione di disinformazione e ai bias. È essenziale che, nel prossimo futuro, vengano adottati regolamenti più rigorosi e sviluppate tecniche avanzate per proteggere i dati e garantire la privacy delle informazioni. Pur riconoscendo i benefici notevoli delle intelligenze artificiali generative, è fondamentale essere consapevoli dei rischi concreti associati al loro uso e affrontarli con la dovuta attenzione e responsabilità.
In aggiunta, questo studio ha delineato un nuovo approccio allo sviluppo, dimostrando come i modelli di grandi dimensioni (LLMs) possano essere applicati a casi d'uso pratici. Un esempio concreto di applicazione open-source è stato presentato, mettendo in luce come l’accessibilità di questi strumenti consenta a chiunque di implementare soluzioni innovative. Tuttavia, questa stessa accessibilità comporta il rischio di utilizzi non leciti. Pertanto, è cruciale non solo valorizzare le potenzialità di queste tecnologie, ma anche affrontare i potenziali abusi e adottare misure adeguate per mitigare tali rischi, bilanciando l'innovazione tecnologica con la necessità di proteggere gli aspetti etici e sociali della nostra vita digitale.
