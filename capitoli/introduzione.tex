\chapter*{Introduzione} %Se si cambia il Titolo cambiare anche la riga successiva così che appia corretto nell'indice
\addcontentsline{toc}{chapter}{Introduzione} %Per far apparire Introduzione nell'indice (Il nome deve rispecchiare quello del chapter)
\pagenumbering{arabic} % Settaggio numerazione normale
Negli ultimi anni, l'esplosione dei Large Language Models (LLMs) ha comportato significativi cambiamenti, sia positivi che negativi. L'obiettivo principale di questa ricerca è valutare l'impatto che i modelli di linguaggio hanno avuto sulla società e fornire una panoramica delle loro diverse applicazioni.\\
Questa analisi permette di comprendere come i modelli di linguaggio possano rappresentare uno strumento utile in ambiti specifici, come ad esempio nella medicina, dove possono migliorare l'efficienza e l'accesso alle informazioni. Tuttavia, è altrettanto importante riconoscere le problematiche associate al loro utilizzo, comprese le potenziali distorsioni, malintesi e abusi che possono derivare da un loro impiego improprio.
In sintesi, la ricerca mira a evidenziare sia i benefici che i rischi legati ai Large Language Models, promuovendo una riflessione critica su come questi strumenti possano essere utilizzati in modo responsabile ed efficace.\\
Nel primo capitolo viene fornita una panoramica sull'Intelligenza Artificiale Generativa, tracciando la sua evoluzione storica fino ai modelli generativi più recenti, sottolineando in modo particolare i modelli di diffusione e i transformers.\\
Il secondo capitolo esplora esempi concreti di Large Language Models, esaminando le loro applicazioni nel mondo del lavoro e le problematiche associate al loro utilizzo. Viene inoltre discussa una serie di tecniche progettate per affrontare e risolvere le problematiche legate alla generazione.\\
Nel terzo capitolo viene presentato un progetto di chatbot sviluppato durante l'esperienza di tirocinio curriculare. Questo progetto illustra l'implementazione di un'architettura Retrieval-Augmented Generation (RAG), dimostrando le potenzialità e le applicazioni pratiche della tecnologia.
